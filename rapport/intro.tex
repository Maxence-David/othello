\section{Introduction}
Le jeu d'othello est un jeu de stratégie qui se joue à deux joueurs, un joueur blanc et un joueur noir. 
Ce jeu ce joue en principe sur un plateau appelé othellier, ce plateau est composé de 64 cases (8*8).
Le but est simple : avoir le plus de pion de sa couleur a la fin de la partie.
Pour cela chaque joueur peut prendre les pion adverse en les encerclant par des pion de ca couleur. 
\\ Le but de notre projet d'algoritmique est de réaliser un  un programme informatique permetant de jouer a l'Othello sur ca machine.
Pour cela il nous faudras réalisé la specification des TAD indispendable a notre futur programme. Puis réalisé la conception preliminaire ainsi que la conception detaillé du probleme.
Dans un second temps nous devront implémenté notre solutiuon en utilisant le language de programmation spécifique à cours le C. 
Pour finir nous devront mettre en plca une batterie de test unitaire pour nous assuré du bon fonctionnement de chaque fonctionnalité implenter.
\\ De plus il nous est imposser de réaliser une Intelligence artificielle dans le but de pouvoir jouer a notre jeu contre l'ordinateur.
Mais aussi de confronté notre algoritme avec celui des autres groupe à l'occasion d'un tournoi.
\\ Ce projet est un premiere approche de projet de groupe pour réaliser un programme informatque, avec tout ce que cela implique de creer un programme du debut a la fin.
Avec nottament la prise de desition sur des choix d'implementation. Mais aussi la gestion du code grace a GIT.  
 Ce projet va  aussi nous permettre  de nous amméliorer dans la programmation en C.
 

