\section{Introduction}
Le jeu d'othello est un jeu de stratégie qui se joue à deux joueurs, un joueur blanc et un joueur noir. 
Ce jeu se joue en principe sur un plateau appelé othellier, ce plateau est composé de 64 cases (8 \times 8).
Le but est simple : avoir le plus de pion de sa couleur a la fin de la partie.\\
Pour cela chaque joueur peut capturer les pion adverse en les encerclant par des pion de sa couleur. 
 Le but de notre projet d'algoritmique est de réaliser un  un programme informatique permetant de jouer a l'Othello sur sa machine.
Pour cela il nous faudra réaliser la specification des TAD indispendable pour notre futur programme. Puis réaliser la conception préliminaire ainsi que la conception detaillée du problème.
Dans un second temps nous devront implémenter notre solution en utilisant le language de programmation spécifique au cours : le C. 
Pour finir nous devront mettre en place une batterie de tests unitaires pour nous assurer du bon fonctionnement de chaque fonctionnalité implentée.\\
De plus il nous est demandé de réaliser une Intelligence artificielle dans le but de pouvoir jouer contre l'ordinateur.\\
Mais aussi de confronter notre algoritme avec celui des autres groupe à l'occasion d'un tournoi.
Ce projet est une première approche de projet de groupe pour réaliser un programme informatque, avec tout ce qui est impliqué par la création d'un programme du debut a la fin.
Avec nottament la prise de decisions sur des choix d'implementation. Mais aussi la gestion du code grâce à GIT.  
 Ce projet va  aussi nous permettre  de nous amméliorer dans la programmation en C.
 

