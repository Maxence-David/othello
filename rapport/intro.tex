\section{Introduction}
Le jeu d'othello est un jeu de stratégie qui se joue à deux joueurs, un joueur blanc et un joueur noir. 
Ce jeu se joue en principe sur un plateau appelé othellier, composé de 64 cases (8 * 8).
Le but est simple : avoir le plus de pions de sa couleur à la fin de la partie.\\
Pour cela chaque joueur peut capturer les pions adverses en les encerclant par des pions de sa couleur. 
 Le but de notre projet d'algorithmique est de réaliser un programme informatique permettant de jouer a l'othello sur sa machine.
Pour cela il nous faudra réaliser la spécification des TAD indispensables pour notre futur programme, puis réaliser la conception préliminaire ainsi que la conception détaillée du problème.
Dans un second temps nous devront implémenter notre solution en utilisant le langage de programmation spécifique au cours : le C. 
Pour finir nous devront mettre en place une batterie de tests unitaires pour nous assurer du bon fonctionnement de chaque fonctionnalité implémentée.\\
De plus il nous est demandé de réaliser une Intelligence Artificielle dans le but de pouvoir jouer contre l'ordinateur, mais aussi de confronter notre algorithme avec celui des autres groupe à l'occasion d'un tournoi.
Ce projet est une première approche du travail de groupe en informatique, avec tout ce qui est impliqué par la création d'un programme du début a la fin, notamment la prise de décisions sur des choix d'implémentation mais aussi la gestion du code grâce à la plateforme GIT.  
 Ce projet va  aussi nous permettre  de nous améliorer dans la programmation en C.
 

